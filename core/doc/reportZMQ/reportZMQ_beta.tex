%%%% OPTION
%% Change class according to your needs
%%  - article (no chapter)
%%  - report
%%  - etc.
\documentclass{article}


\IfFileExists{ifxetex.sty}{%
  \usepackage{ifxetex}%
}{%
  \newif\ifxetex
  \xetexfalse
}
  \ifxetex

\usepackage{fontspec}
\usepackage{xltxtra}
\setmainfont{DejaVu Serif}
\setsansfont{DejaVu Sans}
\setmonofont{DejaVu Sans Mono}
\else
\usepackage[T1]{fontenc}
\usepackage[utf8]{inputenc}
\fi
\usepackage{fancybox}
\usepackage{makeidx}
\usepackage{cmap}
\usepackage{url}
\usepackage{eurosym}

\usepackage[hyperlink]{sysfera}



%%%%
%% TODO:
%%  - ajouter une macro pour mettre l'objet du document
%%  - faire les footers avec l'adresse et tels de SysFera
%%  - enlever un max de paquets docbook


%%%%%%%%%%%%%%%%%%%%%%%%%%%%%%%%%%%%%%%%%%%%%%%%%%%%%%%%%%%%%%%%%%%%%%
%                            CONFIGURATION                           %
%%%%%%%%%%%%%%%%%%%%%%%%%%%%%%%%%%%%%%%%%%%%%%%%%%%%%%%%%%%%%%%%%%%%%%
% Use the following macros to configure your document
%%%%%%%%%%%%%%%%%%%%%%%%%%%%%%%%%%%%%%%%%%%%%%%%%%%%%%%%%%%%%%%%%%%%%%
%%%% OPTION
%% Change language: fr/en
%%  - for French: use french in babel, and fr in \setupsysferalocale
%%  - for English: use english in babel, and en in \setupsysferalocale

\usepackage[french]{babel}
\setupsysferalocale{fr}
\frenchbsetup{CompactItemize=false} % Fix itemize clash 
\usepackage{enumitem}
%%%% OPTION
%% Title and author of the document
\title{Rapport de release ZMQ beta}
\author{K. Coulomb}

%%%% OPTION
%% Document reference
%% Use command \SFdocumentreference to set the document reference.
%% Latter on, you can use the \SFthisdocument macro to retrieve
%% this reference.
\SFdocumentreference{ZMQ beta}
\SFprojectname{VISHNU}
\SFprojectleader{E.P. Capo-Chichi}
%\SFclient{}


%%%% OPTION
%% Release information. If the argument is not empty, then a box with
%% the content of the argument will be visible at the top of the document
%\SFreleaseinfo{Travail en cours} % will show "Travail en cours" at the
                                 % top of the page
\SFreleaseinfo{} % won't show anything

%%%% OPTION
%% Draft watermark. You can also show a grey watermak on all pages of
%% your document using the following command.
%\showwatermark{DRAFT}


%%%% OPTION
%% Add a logo into the header.
%% First parameter sets the width of the image relatively to the
%% \textwidth
%% Second parameter is the path to the image
% \SFsetheadlogo{.25}{fig/logosysfera.pdf}

%%%% OPTION
%% Collaborators:
%% You can redefine the Indexation of the document using the following command:
\renewcommand{\SFindexation}{
  \begin{SFindtable}
    \SFinditem{\writtenby}{K. Coulomb}{5 décembre 2012}
%    \SFinditem{\verifiedby}{Quelqu'un d'autre que l'auteur}{30 septembre 2011}
%    \SFinditem{}{Another Guy}{30 septembre 2011}
%    \SFinditem{\approvedby}{Personne responsable du doc}{30 septembre 2011}
  \end{SFindtable}
}
% \renewcommand{\SFindexation}{} % disable this table


%%%% OPTION
%% Revision History Table:
%% Add a new \SFrevitem entry for adding a new entry in the revision
%% history table.
\renewcommand{\SFrevhistory}{
\begin{SFrevtable}
  \SFrevitem{1}{5/12/2012}{Premiere version de ce document}{K. Coulomb}
\end{SFrevtable}
}
% \renewcommand{\SFrevhistory}{} % disable this table


%%%% OPTION
%% References Table
%% Add a new \SFrefitem entry for adding a new entry in the list of
%% reference documents.
%\renewcommand{\SFreferenceTable}{
%\begin{SFreftable}
%  \SFrefitem{ref1}{techDocument}{Ce document}
%  \SFrefitem{ref2}{techDocument}{Ce document}
%\end{SFreftable}
%}
\renewcommand{\SFreferenceTable}{} % disable this table

%%%% OPTION
%% Authorization Table
%% Add a new \SFauthviewitem entry for adding a new entry in the list of
%% authorized users.
%\renewcommand{\SFauthview}{
%\begin{SFauthviewtable}
%  \SFauthviewitem{SysFera}{Tous les membres de SysFera}
%\end{SFauthviewtable}
%}
\renewcommand{\SFauthview}{} % disable this table
%%%%%%%%%%%%%%%%%%%%%%%%%%%%%%%%%%%%%%%%%%%%%%%%%%%%%%%%%%%%%%%%%%%%%%
%                           /CONFIGURATION                           %
%%%%%%%%%%%%%%%%%%%%%%%%%%%%%%%%%%%%%%%%%%%%%%%%%%%%%%%%%%%%%%%%%%%%%%


\makeindex
\makeglossary


\begin{document}

\frontmatter % do not disable, this is used for page numbering
\maketitle % do not disable, otherwise you won't have any title
%%%% OPTION
\tableofcontents % comment to disable the table of contents
\mainmatter % do not disable, this is used for page numbering


%%%%%%%%%%%%%%%%%%%%%%%%%%%%%%%%%%%%%%%%%%%%%%%%%%%%%%%%%%%%%%%%%%%%%%
%              Write your document below this comment                %
%%%%%%%%%%%%%%%%%%%%%%%%%%%%%%%%%%%%%%%%%%%%%%%%%%%%%%%%%%%%%%%%%%%%%%

\section{Vocabulaire}

\begin{tabular}{|l|l|}
\hline
dispatcher & Elément optionnel de la version ZMQ servant d'annuaire si le client ne connait pas tout les serveurs \\
\hline
ZMQ & Librairie de passage de message \\
\hline
DIET & Middleware en corba permettant de lier des clients à des serveurs \\
\hline
VISHNU & Logiciel open source permettant à des clients d'avoir des services prédéfinis d'effectués \\
\hline
\end{tabular}

\section{Pr\'esentation}

L'objectif de ce document est de présenter l'état de la branche
ZMQ du projet VISHNU. Il contient des comparatifs de performance,
des résultats de tests, des remarques sur l'état d'avancement.
Les comparaisons se font principalement par rapport à la branche
basée sur DIET.

\section{Comparatif de performance}

\subsection{Configuration}
Les tests de performance suivant ont été réalisés sur une seule
machine dual core hyperthreadé sous l'environnement ubtuntu 11.10
entre la release VISHNU beta2 et l'état de la branche ZMQ en 
décembre 2012. Sur ces tests, une seule machine contenait tout 
les processus nécessaires. Les tests ont été effectués les uns
à la suite des autres et non pas en parallèle pour éviter tout
conflit. \\
Le parallélisme a été simulé avec 4 scripts de 250 connexions séquentielles lancés en parallèle.

\begin{tabular}{|c|c|}
\hline
OS & Ubuntu 11.10 \\
\hline
Compilateur & gcc 4.6 \\
\hline
DIET & 2.8.1 \\
\hline
ZMQ & 2 \\
\hline
\end{tabular}

\subsection{R\'esultats}

\begin{tabular}{|c|c|c|c|}
\hline
 X & DIET & DISPATCHER & DIRECT \\
\hline
1000 connect seq & real : 14m8s & real : 13m26 & real : 13m44s \\
 & user : 9m51s & user : 9min27s & user : 9m8s \\
 & syst : 0m14s & syst : 0m7s & syst : 0m8s \\
\hline
1000 connect par & real : 10m04s & real : 7m33s & 7m48s \\
\hline
\end{tabular}

\subsection{Analyse}
De manière surprenante, la partie ZMQ avec dispatcher a été plus rapide que
 la partie sans dispatcher. Les temps ont été mesuré avec \textit{time} 
en séquentiel et en se basant sur la première et la dernière session 
ouverte en parallèle. Les temps en eux mêmes ne sont pas significatifs,
tout tournait sur la même machine ou d'autres processus prenaient des
ressources, mais le nombre de process externe et leur impact était stable
 lors des diverses expériences.

\section{Test stabilité}
La version ZMQ a tourné pendant plusieurs jours d'affilé sans aucun souci.
Seul un cas potentiellement problématique s'est levé, lors d'un test tout
un week end, le serveur UMS était passé en arrêt dans supervisord et ne 
tournait plus alors qu'il aurait du rester à tourner.
Lors de ce test, le serveur était chargé d'une quinzaine de requêtes
simultanées en permanence suivant un scénario prédéfini 
(connexion, service liste machine, service liste sessions, déconnexion).

\section{Test de charge}
\subsection{Tests}
Le test de charge correspond à une centaine d'utilisateurs en simultané
qui font des requêtes (référence bug 448 bugzilla).

\begin{tabular}{|c|c|}
\hline
DIET & Voir Bug 448 \\
\hline
DISPATCHER & ralentissement \\
\hline
DIRECT & OK \\
\hline
\end{tabular}

\subsection{Analyse}
L'aspect lenteur était lié au fait que l'on ne pouvait pas traiter les 100
requêtes à la fois et que les timeout étaient court (2 secondes), les premières
 passaient bien, mais une fois que les messages n'avaient pas de réponse dans
le délai, la procédure de renvoyer le message était mise en route, ce qui a 
entretenu la surcharge du système et un nombre important de messages inutiles.
L'augmentation du timeout a amélioré les choses (le timeout au niveau du client
ET le timeout au niveau du dispatcher), mais cela n'empêche qu'il fallait un 
certain temps entre certaines requêtes et le moment ou l'on avait la clé de
session. \\
Pour améliorer cela, un correctif permettant de modifier la taille du pool de 
threads de communication a été mis en place.

\section{Consommation}

\subsection{Test}
Ce test a pour but de rechercher des fuites mémoires. L'idée est de laisser
tourner VISHNU et d'observer la consomation mémoire. Le serveur utilisé pour
l'expérience est UMS. Dans une machine virtuelle sous kubuntu, on a lancé
UMS dans un conteneur lxc et observé les consomation avec la commande 'top'. Dans le tableau, la colonne temps correspond au temps depuis lequel 
le processus umssed a été lancé, en heure. La dernière colonne usage 
correspond au fait que la plateforme était utilisée ou non lors de la
période. Le reste des données correspond à la sortie de la commande top.

\begin{tabular}{|c|c|c|c|c|c|c|c|c|c|c|c|c|c|}
\hline
TEMPS & PID & USER & PR & NI & VIRT & RES & SHR & S & CPU & MEM & TIME+ & COMMAND & usage \\
\hline
0 & 363 & root & 20 & 0 & 37780 & 10m & 6668 & S & 0.0 & 2.1 & 0:00.12 & umssed & NON \\                      
0 & 365 & root & 20 & 0 & 18732 & 4412 & 3004 & S & 0.0 & 0.9 & 0:00.01 & umssed  & NON \\
\hline
2 & 363 & root & 20 & 0 & 129m & 35m & 2128 & S & 0.0 & 7.2 & 2:16.22 & umssed & OUI\\
2 & 365 & root & 20 & 0 & 18732 & 1496 & 884 & S & 0.0 & 0.3 & 0:02.02 & umssed & OUI \\
\hline
3 & 363 & root & 20 & 0 & 135m & 27m & 2568 & S & 0.0 & 5.6 & 2:52.71 & umssed & OUI \\
3 & 365 & root & 20 & 0 & 18732 & 2556 & 1876 & S & 0.0 & 0.5 & 0:10.69 & umssed & OUI \\
\hline
4 & 363 & root & 20 & 0 & 135m & 27m & 2568 & S & 0.0 & 5.6 & 2:52.71 & umssed & NON \\
4 & 365 & root & 20 & 0 & 18732 & 2556 & 1876 & S & 0.0 & 0.5 & 0:11.44 & umssed & NON \\
\hline
19 & 363 & root & 20 & 0 & 135m & 8496 & 1728 & S & 0.0 & 1.7 & 2:52.71 & umssed & NON \\
19 & 365 & root & 20 & 0 & 18732 & 2396 & 1844 & S & 0.0 & 0.5 & 0:32.64 & umssed & NON\\
\hline
24 & 363 & root & 20 & 0 & 135m & 8496 & 1728 & S & 0.0 & 1.7 & 2:52.71 & umssed & NON \\
24 & 365 & root & 20 & 0 & 18732 & 2396 & 1844 & S & 0.0 & 0.5 & 0:40.37 & umssed & NON \\
\hline
27 & 363 & root & 20 & 0 & 135m & 4220 & 0 & S & 0.0 & 0.8 & 2:52.71 & umssed & NON \\
27 & 365 & root & 20 & 0 & 18732 & 1100 & 560 & S & 0.0 & 0.2 & 0:45.36 & umssed & NON \\
\hline
\end{tabular}

\subsection{Analyse}
Il existe toujours un problème de fuite mémoire dans VISHNU version ZMQ.
En effet pendant la longue inactivité de la nuit, la mémoire physique
utilisée est passée de 27m à 8496m avant de rebaisser un peu par la suite
à 4220m.


\section{Futures orientations}

\subsection{Conteneur de service}
Le but de cette amélioration sera de faire un conteneur
de service, sur une machine il n'y aura plus divers binaires (FMS, UMS, ...) mais un
seul binaire qui pourra charger comme des plugins les services de différents modules.
Ce conteneur ne pourra charger que les plugins qui ont été compilés sur la machine. \\
Les impacts sont la gestion des processuus (arrêt, démarage, lister).\\
Avantages : Limitera la duplication des tunnels SSH entre les machines, limitera le
nombre de processus, simplifiera le déploiement (moins d'éléments).

\subsection{Protocole de communication}
Le but sera d'ajouter des informations aux messages et de modifier ceux qui réceptionnent
pour éviter que lorsqu'un message est renvoyé, le serveur ne le retraite (cas des timeout
ou le service est effectué plusieurs fois à cause de cela). \\
Les impacts sont sur la sérialisation/désérialisation et la manière de traiter les messages
(les messages envoyés ne sont pas uniformes). \\
Piste : Encapsuler dans un objet message, uniquement préfixer les messages, ... \\
Avantages : Supprime un bug ou la même commande utilisateur peut donner plusieurs 
exécutions de cette commande coté serveur

\subsection{Robustifier/améliorer}
Le but sera d'améliorer les cas d'erreur et la gestion des pannes (rejoint en partie
le coté protocole de communication). Cela comprend également une partie nettoyage mémoire
concernant les fuites qui ont été appercues dans les tests. Une partie amélioration des
complexité cyclomatiques et amélioration des performances peut être incluse.\\
Piste : valgrind, electric fence, gprof \\
Avantage : Amélioration des performances, diminution de l'impact sur la machine

\subsection{Unifier les fichiers de configuration}
Le but sera de préfixer les variables des fichiers de configuration liées pour n'avoir
qu'un seul fichier de configuration que l'on passera à tout le monde.\\
Impact : Faible, renomer dans core et dans les binaires les variables, 
modifier les exemples de fichier de configuration \\
Avantage : Plus qu'un seul fichier pour toute la plateforme que l'on peut dupliquer.

%%%%%%%%%%%%%%%%%%%%%%%%%%%%%%%%%%%%%%%%%%%%%%%%%%%%%%%%%%%%%%%%%%%%%%
%                            BIBLIOGRAPHY                            %
%%%%%%%%%%%%%%%%%%%%%%%%%%%%%%%%%%%%%%%%%%%%%%%%%%%%%%%%%%%%%%%%%%%%%%
%%%% OPTION
%% You can deactivate bibliography by commenting the following 3
%% lines.
%% If you wish to add a bibliography, you need to update the
%% \biblography command. Its argument is the name of .bib file.
\clearpage
%\bibliography{techDocument}
%\bibliographystyle{plain}
%%%%%%%%%%%%%%%%%%%%%%%%%%%%%%%%%%%%%%%%%%%%%%%%%%%%%%%%%%%%%%%%%%%%%%
%                           /BIBLIOGRAPHY                            %
%%%%%%%%%%%%%%%%%%%%%%%%%%%%%%%%%%%%%%%%%%%%%%%%%%%%%%%%%%%%%%%%%%%%%%
    
\end{document}
